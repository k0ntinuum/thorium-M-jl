

\documentclass{article}
\usepackage[utf8]{inputenc}
\usepackage{setspace}
\usepackage{ mathrsfs }
\usepackage{graphicx}
\usepackage{amssymb} %maths
\usepackage{amsmath} %maths
\usepackage[margin=0.2in]{geometry}
\usepackage{graphicx}
\usepackage{ulem}
\setlength{\parindent}{0pt}
\setlength{\parskip}{10pt}
\usepackage{hyperref}
\usepackage[autostyle]{csquotes}

\usepackage{cancel}
\renewcommand{\i}{\textit}
\renewcommand{\b}{\textbf}
\newcommand{\q}{\enquote}
%\vskip1.0in





\begin{document}

{\setstretch{0.0}{


\b{Thorium-M} is a specialized binary version of Cesium, and it is a simpler variant of the Thorium idea. It uses $n \times n$  matrix of bits as a key. The trace of the matrix is added (mod 2) to the plaintext bit to get the ciphertext bit. The key matrix is adjusted as a function of the plaintext bit. In Thorium-m, only one row or column is adjusted per bit of plaintext, so Thorium-m is faster, without any obvious loss of security. 

Here's the encoding function: 


\begin{verbatim}
spin_row(k,i) = circshift(k[i,:], k[i,i]  + 1)
spin_col(k,i) = circshift(k[:,i], k[i,i]  + 1)

function encode(p,q)
    k = copy(q)
    c = Bool[]
    for i in eachindex(p)
        push!(c,Bool((tr(k) + p[i])%2))
        if  Bool(p[i]) 
            k[mod1(i,n),:] = spin_row(k,mod1(i,n))
        else 
            k[:,mod1(i,n)] = spin_col(k,mod1(i,n))
        end
    end
    c
end

\end{verbatim}

The autospin function of the \q{m} version of Thorium is also gentler and less expensive. It uses bits from the key (column by column) to update itself as if by plaintext bits, one column at a time. 

\begin{verbatim}


function autospin(q,c)
    k = copy(q)
    for i in 1:n Bool(q[i,c]) ? k[i,:] = spin_row(k,i) : k[:,i] = spin_col(k,i) end
    k
end

\end{verbatim}


Note that all shifts are by one unit, so the plaintext determines only the order of the shifts ( such as \q{row, row, col} versus \q{col, row, col} and so on.)

The point of \b{autospin} is to meet each round with a different version of the key. Decryption requires using these keystates in reverse order. 

\begin{verbatim}

function encrypt(p, q, r)
    for i in 1:r
        k = autospin(q, mod1(i,n))
        p = encode(p,k)
        p = reverse(p)
    end
    p
end


function decrypt(p, q, r)
    for i in 1:r
        k = autospin(q,mod1(r + 1 - i,n))
        p = reverse(p)
        p = decode(p,k)
    end
    p
end

\end{verbatim}

The \q{official} (ideal) version sets the rounds equal to $n$. This gives us a well defined function $f_k$ with only $k$ for a parameter. Likewise we have $f^{-1}_k$.


}}
\end{document}
